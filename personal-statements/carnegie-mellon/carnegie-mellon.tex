\documentclass[11pt,letterpaper]{letter}

\usepackage{geometry}
\geometry{letterpaper, margin=1.5in}

\usepackage{enumitem}
\setlist{nosep} % or {noitemsep}

\usepackage{dirtytalk}

% - I am a strong candidate for the PhD program at Carnegie Mellon
%   - I have previous research experience
%     - I have in-domain experience that is relevant
%     - I have been made aware of and developed the skills for research
%       - reading technical papers
%       - developing hypotheses and designing experiments to validate them
%       - supporting assumptions with data
%       - collaborating with an advisor
%         - developed my communication skills
%   - I am exceptionally self-motivated
%     - I designed, developed and released an app to the OSU student body over the course of several months with no extrinsic motivation
%     - I am turning down a job offer from Microsoft to pursue research
%   - I am a strong programmer
%     - I have four professional experiences writing code
%     - I won three hackathons at OSU     
%   - I am especially interested in Carnegie Mellon
%     - Dr Tom Mitchell's work is my area of interest

\begin{document}

I'm applying to the PhD program at Carnegie Mellon University and I argue that I'm an exceptionally strong candidate because of my previous research experience, exceptional motivation, and strong programming background.
I'm getting a graduate degree because I know I want to pursue a career in natural language processing research after graduation.

My primary area of research interest in natural language processing (NLP) with a focus on machine comphrension.
My undergraduate thesis applies pretrained language models to a sentence classification task and analyzes the models' parameters to understand how they adapt to different editing tasks.
Working on my thesis has taught me skills not emphasized in my classwork; specifically, how to:

\begin{enumerate}
  \item Read technical papers to apply their learning to my work.
  \item Develop hypotheses and then design experiments to validate them.
  \item Support my assumptions with data and analysis.
  \item Collaborate with an advisor and clearly communicate my ideas.
\end{enumerate}

Beyond these \say{meta}-skills, working on a project in my area of interest has given me a taste of the field's current work and existing problems. 
Needless to say, my experience has only made me more excited about the current state of the field and how I can contribute.

Besides my research experience, I have also demonstrated the self-motivation and independent, disciplined workstyle necessary for graduate work.
In the spring of 2018, I created a mobile app with three other students to help Ohio State students resell football tickets.\footnote{https://salty.software/ticketbay}
By the fall of 2019, we had transferred two seasons of tickets valued over \$165K between more than 7K users, without any external academic or financial motivation.
Understanding the discipline and long-term thinking required to turn an idea into reality is another facet of graduate work for which I am especially well-suited.

In addition to research experience and exceptional self-motivation, my intrinsic passion for NLP makes me a strong candidate.
Let me explain: I'm a great computer science student.
I've won three hackathons at OSU, two of which have 800+ participants (largest in Ohio).
I've interned for three summers and had a full-time software engineering offer from Microsoft.
Throughout these experiences, my thesis work has been unbelieveable more interesting and invigorating than my software engineering experience.
I'm turning down more money than I've collectively earned over my life to go get an advanced degree in NLP because my research work has been more interesting than classes; than freelance developing; than (paid) summer internships.
I \textit{really} want to work in NLP research, and I'm making a deliberate decision to pursue that dream.

That's what makes me special to Carnegie Mellon. 
What makes Carnegie Mellon special to me is the abundance of smart people working on hard problems that I find interesting. 
One such problem is understanding how current neural models work in comparison to human brains.
At CMU, Dr. Mitchell's Brain Image Analysis group and their research comparing pretrained neural language models with the human brain is a problem that I find extremely interesting.
Understanding these relationships will help us understand how neural models actually work, making it possible to continue improving them.
Comparing how humans and artificial models think about editing text would further our understanding, and my experience in modeling edits with neural models would be invaluable.

In short, I'm really excited about NLP research, I have the fundamental skills to be successful, and Carnegie Mellon has opportunities where I can have a big impact.

\end{document}